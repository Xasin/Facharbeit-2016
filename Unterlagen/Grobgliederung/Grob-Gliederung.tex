%Papierformat & Schriftgröße
\documentclass[a4paper, 11pt]{article}

%Silbentrennung & Umlaute
\usepackage[ngerman]{babel}
\usepackage[utf8]{inputenc}

\usepackage[left=3.00cm, right=3.0cm, top=3.00cm, bottom=4.00cm]{geometry}

%Überschriften auf alle Seiten
\pagestyle{headings}

\begin{document}


%Eigentliche grobe Gliederung
\section[Thema]{Thema der Facharbeit}
\subsection[Übersicht]{Übersicht über das Thema}
Das Thema meiner Facharbeit wird \emph{Der Einfluss des 3D-Druckes auf unseren Alltag} lauten.
Sie wird sich, wie im Titel bereits nahe gelegt, mit dem Einfluss der momentan rapide ablaufenden Entwicklung des 3D-Druckes befassen.

Fokus wird hierbei nicht nur auf die technischen Details gelegt, sondern ebenfalls auf die für uns direkt bemerkbaren Einflüsse der neuen Technologien auf alltägliche Dinge. 

Nur um einige zu nennen: 
\begin{itemize}
\item Chirurgie
	\begin{itemize}
	\item Zahnmedizin
	\item Maßgefertigte Prothesen
	\item Künstliche Spenderorgane
	\end{itemize}
\item Industrie
	\begin{itemize}
	\item Billigere Spezialanfertigungen / Kleinserien
	\item Revolutionierung des Entwicklungsprozesses neuer Bauteile
	\item Anfertigung komplexer Baustücke mit weniger Aufwand (z.~B. Motoren oder ganze Karosserien)
	\end{itemize}
\item Lifestyle 
	\begin{itemize}
	\item Möbel	
	\item Essen (Verzierungen, gedruckte Backwaren)
	\item Schmuck (Individualisierte Geschenke)
	\item Hobby (Durch DIY-3D-Drucker\footnote{DIY = Do It Yourself. Es handelt sich hierbei oft um Drucker, welche mit billigen Teilen und eigener Arbeit selbst zusammen gebaut werden können. Diese sind meist billiger bei gleicher Funktionalität, weshalb sie sich gut für Einzelpersonen eignen})
	\end{itemize}
\end{itemize}

\subsection{Motivation}
Ich habe das Thema dieser Facharbeit aus persönlichem Interesse, jedoch auch wegen der sehr aktuellen Situation des Themas gewählt. Ich finde, die Entwicklung neuer Technologien wird durchaus vieles in unserem Leben ändern; Dieses Geschehen liegt momentan jedoch noch sehr im Unbekannten und sollte meiner Meinung nach genauer betrachtet werden.
Zusätzlich habe ich die Möglichkeit, sowohl durch Kontakte bei Pöppelmann als auch durch eigene Erfahrungen mit 3D-Druckern und dessen Potenzial die Facharbeit mit guten Argumenten aus zu statten.

Somit finde ich dass die Wahl des Themas sich für mich als sehr passend erwiesen hat, und hoffe darauf dass es sich als sehr interessant erweist.


\section[Aufbau]{Aufbau der Facharbeit}
\subsection{Einleitung}
Die Einleitung meiner Facharbeit wird sich, ähnlich wie in der Präsentation des Themas in dieser Zusammenfassung, damit befassen dass es viele Punkte gibt welche vom 3D Druck beeinflusst werden. 

Zunächst wird das Thema dem Leser näher gebracht, hierzu folgen einige kurze Erläuterungen warum das Thema gewählt wurde etc.
Hiernach wird genauer festgelegt, womit sich die Facharbeit befasst, namentlich nur den \emph{Auswirkungen} des 3D-Druckes, nicht aber wie genau die verschiedenen Verfahren arbeiten\footnote{Es wird hier jedoch, je nach genaueren Details, kurz auf die Verfahren an sich eingegangen, bzw. wann/wie/wofür sie entwickelt wurden.}.

Schließlich wird eine zentrale Fragestellung festgelegt, die im Laufe der Facharbeit aufgeklärt werden sollte. Dem Titel der Facharbeit folgend wird diese sich damit befassen, was wir in den kommenden Jahren an Neuerungen aus dem Bereich 3D-Druck erwarten können. 

\subsection{Hauptteil}
Der Hauptteil der Facharbeit wird sich nun gemäß der Fragestellung mit dem Thema 3D-Druck auseinander setzten.
Anfangs wird er einen groben Überblick über den momentanen Stand der Technologie geben; Schwächen aber auch Stärken werden aufgelistet und miteinander verglichen.

Darauf folgt nun ein Ausblick auf mögliche Entwicklung der Technologie des 3D-Druckes, wie sich die Effizienz der Geräte, vor allem im Vergleich zu momentanen Produktionsmöglichkeiten, ändern wird, aber auch was für neue Möglichkeiten zukünftig zur Verfügung stehen könnten.
Dies wird eventuell zusätzlich noch durch Aussagen der Firma Pöppelmann belegt, in wie fern sie ihre Produktion durch 3D-Druck erweitern wollen, und was ihre Einschätzungen zur Entwicklung sind.

\subsection{Schlussteil}
Der Schlussteil fasst nun die im Hauptteil gesammelten Ergebnisse zusammen. Hierbei werden einerseits Vergleiche des momentanen Zustandes zu den vermuteten Entwicklungen angestellt, andererseits werden auch eventuelle unerwartete Ergebnisse oder Ereignisse aufgelistet.

Zuletzt wird dann aus den gesammelten Ergebnissen gemäß der anfänglich formulierten Frage eine konkrete Antwort entwickelt; Vermutungen über die zukünftigen Änderungen in unserem Alltag werden, belegt durch diese Informationen, aufgelistet.

\end{document}