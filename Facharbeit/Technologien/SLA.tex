\subsubsection{Stereolithographie (SLA)}

Die Stereolithographie ist eine alternative Technologie zur Erstellung von 3D Modellen, welche mit hohem Detail und hoher Geschwindigkeit arbeiten kann. Jedoch sind die Modelle lichtempfindlich, und das Ausgangsmaterial ist um ein vielfaches teurer als bei anderen Varianten.

Beim SLA Verfahren wird, anders als bei anderen Möglichkeiten, mit einem bei Raumtemperatur flüssigem Kunstharz gearbeitet, welches sich in einem Becken befinden.
Dieses Kunstharz wird nun mithilfe eines starken UV-Lasers an den gewollten Stellen gehärtet, wobei ähnlich wie beim FDM Verfahren schichtweise gearbeitet wird.

Ist eine Schicht des Objektes fertig, so wird entweder der Träger auf dem sich das Modell befindet ein Stück weiter aus dem Kunstharz heraus gezogen (Hierbei trifft der Laser von unten durch eine durchsichtige Platte auf das Kunstharz) oder eine kleine Menge Harz hinzu gegeben, sodass das Modell konstant mit einer Schicht dieses Materials bedeckt ist (Hierbei trifft der Laser von oben auf das Harz).

Ist das Modell fertig kann es aus dem Harz entnommen werden, muss allerdings oftmals noch mithilfe eines UV-Schrankes weiter gehärtet werden, um die Stabilität des Modells zu gewährleisten.

Sehr vorteilhaft ist hierbei dass die Modelle mit hoher Präzision gedruckt werden können da der Laser quasi unbegrenzt fein sein kann, wodurch kaum noch das Objekt nachbearbeitet werden muss. Höchstens die für Überhänge nötigen Stützstrukturen müssen entfernt werden, diese können jedoch oftmals um ein Vielfaches kleiner sein da das Modell während des Druckes vom Kunstharz umgeben ist welches die gleiche Dichte besitzt, und so \textquotedblleft in der Schwebe\textquotedblright ~gehalten wird.

Nachteile des SLA sind allerdings vor allem hohe Kosten des Kunstharzes, sowie die verbleibende Lichtempfindlichkeit des Modells, welche bei übermäßigem Lichteinfall zu Sprödigkeit führen kann. \parencite[Informationen aus:][]{DRUCKVERFAHREN}