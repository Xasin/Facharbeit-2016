Es steht mittlerweile außer Frage dass 3D Druck einen gravierenden Einfluss auf unser Leben hat und haben wird. Viele neue Technologien sind schon in der Entwicklung, und sind kurz davor für die Industrie verfügbar zu sein. Viele Konzerne planen bereits 3D Druck in der Produktion an zu wenden, oder verwenden sie bereits für rapid Prototyping.


\subsection{Industrie}

3D Druck in der Industrie ist jetzt schon ein wichtiger Bestandteil, vor allem für kleinere Firmen, aber auch für die Entwicklung neuer Produkte in größeren Konzernen. Einige Firmen haben Konzepte jetzt schon in der Entwicklung, oder verwenden bereits gedruckte Teile um ihre Produktion zu verbessern.

So plant Airbus z.~B., mithilfe von 3D Druckern mehrere wichtige Teile für ihre Flugzeugproduktion zu verwenden, wodurch die Effizienz gesteigert und kosten gesenkt werden könnten. Die Firma hat sich bereits mit anderen Konzernen an der Entwicklung der nötigen Technologien gesetzt, und ko-operiert z.~B. mit 3D Druck Firma Stratasys, welche für sie bereits im letzten Jahr für ein Flugzeug über 1 000 Teile gedruckt hat, welche in der Produktion für Abdeckungen, Mechanik und andere Teile verwendet wurden. Die gedruckten Bauteile erfüllten den kompletten Standard der Luftfahrt, und waren zudem durch die Bauweise passgenau, leicht und stabil \parencite{1000_PART_PLANE}.

Airbus gab zudem letztens bekannt dass sie sich mit \company{Local Motors} zusammen legen würden, um weiter im Bereich des 3D Druck zu Forschen. Hierfür gab der Konzern ein 150 000 000\EURO Kapital bekannt, welches für die Entwicklung und Forschung bereit stehe. Zudem haben sie bereits einen der von \company{Additive Industries} entwickelten \emph{MetalFAB1} in ihre Produktionslinie eingebaut, und planen darauf schon in den nächsten Jahren mehrere Tonnen Bauteile mithilfe von 3D Druckern so zu produzieren \parencite{AIRBUS_METALFAB}.

Auch einige am Käufer 'näher' liegende Firmen beteiligen sich bereits an der Entwicklung der Produktion mithilfe von 3D Druckern.
So z.~B. hat \company{Adidas} letztens bekannt gegeben dass sie eine spezielle Schuhsole, sog. \emph{Futurecraft Soles}, bereitstellen wollen. Diese speziellen Schuhsohlen werden aus einem Scann eines Fußes generiert, und mithilfe von 3D Druckern produziert. So könnten diese Sohlen für jede Fuß passgenau hergestellt werden, und könnte zudem für bestimmte Aufgaben den Fuß an den passenden Stellen unterstützten \parencite{ADIDAS}. Eine andere Firma, \company{New Balance}, setzt ebenfalls auf eine ähnliche Technologien, und will selbst individualisierte Sohlen, welche an den Fuß des Benutzers angepasst werden, an bieten \parencite{NEW_BALANCE}.

Es ist somit klar ersichtlich dass 3D Druck bald schon in der Industrie eine große Rolle spielen wird. Viele komplexe Bauteile können sehr leicht hergestellt werden, Einzelteile können auf Knopfdruck fabriziert werden, und Individualisierungen von Produkten werden nicht mehr sein als eine kleine Eingabe am Computer. Es wird eine deutliche Revolutionierung des Entwicklungs-, aber auch des Herstellungsprozesses geben, welche sich stark auf die Qualität der Teile und Systeme auswirken wird.
\subsection{Medizin}

Kaum ein Feld könnte mehr vom 3D Druck profitieren als die Medizin. Da jeder Patient individuelle Betreuung benötigt ist es beinahe unmöglich mit konventionellen Methoden passenden Bauteile her zu stellen. 3D Drucker jedoch können aus fast jedem verfügbaren Material eine beliebige, leicht änderbare Form herstellen. Dies sorgt z.~B. für viel bessere Prothesen und Implantate, welche auf den Endnutzer abgestimmt werden können, und zudem um ein vielfaches schneller verfügbar sind und stark gesenkte Kosten haben. So wurden jetzt schon 3D gedruckte Implantate verwendet, welche mit hoher Komplexität und in nur einem Bruchteil der normalen Produktionszeit hergestellt werden konnten, wie z.~B. ein aus Titan angefertigter, mit komplexen Strukturen versehener Kiefer, welcher in nur wenigen Stunden gedruckt werden konnte \parencite{IRONJAW}.

3D gedruckte Modelle können allerdings nicht nur als Prothesen und Implantate verwendet werden. Chirurgen können z.~B. einen Scan eines komplexen Knochenbruches nehmen, und diesen als 3D Modell drucken. So können noch vor der eigentlichen Operation wichtige Handgriffe geübt bzw. geplant werden, und können so die eigentliche Operation einfacher verlaufen lassen \parencite{ORTHOPEDICS}.

3D Druck kann jedoch mehr als das. Wie vorhin bereits erläutert sind Bio-Drucker in der Entwicklung. Diese Drucker könnten die Medizin revolutionieren. Anstelle eines Spenderorgans welches von einem passenden Spender kommen muss, eine Suche die oftmals sehr lange dauert und viele Probleme mit sich bringt wie z.~B. dass der Empfänger das Organ abstoßt, könnte man bald Zellen aus dem Empfänger entnehmen und sie mithilfe einiger Prozesse zu einem passenden Organ 'drucken'. Diese neue Organ wäre beinahe 100\% kompatibel mit dem Empfänger, und könnte in exakt die passende Form und Größe gebracht werden. Zudem wären solche Organe immer in einem absehbaren Zeitraum verfügbar, wodurch eventuelle lebensrettende Methoden genauer eingeplant werden könnten \parencite{ORGANOVO}.

Diese Technologien hätten jedoch nicht nur für die Chirurgie einen wichtigen Einfluss. Auch die pharmazeutische Industrie könnte davon stark profitieren: 3D Gedruckte Organe, welche hergestellt werden könnten ohne dabei Menschen oder Tiere zu verletzten, wären perfekt für die Verwendung als Test-Organe geeignet, ohne dabei Leben zu gefährden, sollte die Medizin Nebenwirkungen aufweisen. So wäre es viel schneller und vor allem fehlerfreier möglich, eine neue Medizin aus zu testen \parencite{ORGANOVO}. 

Es wird sehr deutlich dass 3D Druck die gesamte Chirurgie verändern könnte. Von der Medizin die verwendet wird über die Organe und Implantate die zur Verfügung stehen bis hin zu den eigentlichen Instrumenten der Chirurgen werden diese neuen Technologien einen starken Einfluss besitzen, und vieles kostengünstiger, individueller, schneller und sicherer machen können.

\newpage
\subsection{Privatgebrauch}

Auch der private Bereich wird durch 3D Drucker stark geformt und verändert werden, hauptsächlich jedoch durch die Änderungen in der Industrie, Medizin und anderen Gebieten. Produkte, Operationen, Kleidung, Geschenke und vieles mehr wird dank der neuen Technologien um ein vielfaches günstiger individualisiert werden können.

Doch auch der Privathaushalt selbst wird stark von günstigen 3D Druckern profitieren können: Ist einmal etwas kleines kaputt gegangen, so kann es leicht selbst gedruckt und repariert werden. Auch können z.~B. Eltern mithilfe eines Druckers Spielzeug günstig für ihre Kinder herstellen. Oder aber die Kinder selbst können mithilfe von 3D Druckern ihrer Kreativität freien Lauf lassen, und eigene Objekte erstellen und betrachten. 

Auch der schulische Bereich könnte vom 3D Druck profitieren. So könnten die Drucker z.~B. im Kunstunterricht die Modelle der Schüler produzieren, oder je nach spezifischem Gebrauch der Lehrer anatomische Modelle fabrizieren.

Ein bestes Beispiel hierfür ist ein Komplettset zum Bau eines 3D Druckers, welches von Fischertechnik für 700\EURO diesen Sommer heraus kommen wird. Das Set enthält alles, was zum Bau eines Druckers nötig ist, und erlaubt es so jedem sich selbst spielend leicht einen 3D Drucker zu bauen und über dessen Funktionsweise zu lernen \parencite{FISCHERTECHNIK}.

Jedoch wird es trotz all diesen durchaus nützlichen Anwendungen vermutlich nicht, wie manche sagen, in jedem Haushalt einen 3D Drucker geben. Selbst bei billigeren Geräten würden diese einfach kaum genutzt, eine Anschaffung würde sich nicht lohnen. Vielmehr wird es eher 3D druck \textquotedblleft Copy-Shops\textquotedblright ~geben, welche den Service des 3D Druckens eines Modells an bieten, bzw. die bereits bestehenden Online-Dienste könnten an Wichtigkeit gewinnen.

3D Druck wird dementsprechend auch für den Endnutzer direkt einige wichtige Neuerungen mit sich bringen. Diese jedoch werden meistens in der Industrie selbst ihren Platz finden, und deshalb nur indirekt für uns relevant sein.