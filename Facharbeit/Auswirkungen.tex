Es steht mittlerweile außer Frage dass 3D Druck einen gravierenden Einfluss auf unser Leben hat und haben wird. Viele neue Technologien sind schon in der Entwicklung, und sind kurz davor für die Industrie verfügbar zu sein. Viele Konzerne planen bereits, 3D Druck in der Produktion an zu wenden, oder verwenden sie bereits für das Prototyping.


\subsection{Industrie}

3D Druck in der Industrie ist jetzt schon ein wichtiger Bestandteil, vor allem für kleinere Firmen, aber auch für die Entwicklung neuer Produkte in größeren Konzernen. Einige Firmen haben Konzepte jetzt schon in der Entwicklung, oder verwenden bereits gedruckte Teile um ihre Produktion zu verbessern.

So plant Airbus z.~B. darauf, mithilfe von 3D Druckern mehrere wichtige Teile für ihre Flugzeugproduktion zu verwenden, wodurch die Effizienz gesteigert und kosten gesenkt werden könnten. Die Firma hat sich bereits mit anderen Konzernen an der Entwicklung der nötigen Technologien gesetzt, und ko-operiert so mit 3D Druck Firma Stratasys, welche für sie bereits im letzten Jahr für ein Flugzeug über 1 000 Teile gedruckt hat, welche in der Produktion für Abdeckungen, Mechanik und andere Teile verwendet wurden. Die gedruckten Bauteile erfüllten den kompletten Standard der Luftfahrt, und waren zudem durch die Bauweise passgenau, leicht und stabil \parencite{1000_PART_PLANE}.

Airbus gab nun letztens bekannt dass sie sich mit \company{Local Motors} zusammen legen würden, um weiter mit 3D Druck zu Forschen. Hierfür gab der Konzern ein 150 000 000\EURO Kapital bekannt, welches für die Entwicklung und Forschung bereit stehe. Zudem haben sie bereits einen der von \company{Additive Industries} entwickelten \emph{MetalFAB1} in ihre Produktionslinie eingebaut, und planen darauf schon in den nächsten Jahren mehrere Tonnen Bauteile mithilfe von 3D Druckern so zu produzieren \parencite{AIRBUS_METALFAB}.

\vspace{7pt}

Auch einige am Käufer 'näher' liegende Firmen beteiligen sich bereits an der Entwicklung der Produktion mithilfe von 3D Druckern.
So z.~B. hat \company{Adidas} letztens bekannt gegeben dass sie eine spezielle Schuhsole, sog. \emph{Futurecraft Soles}, bereitstellen wollen. Diese speziellen Schuhsohlen werden aus einem Scann eines Fußes generiert, und mithilfe von 3D Druckern produziert. So könnten diese Sohlen für jede Fuß passgenau hergestellt werden, und könnte zudem für bestimmte Aufgaben den Fuß an den passenden Stellen unterstützten \parencite{ADIDAS}. Eine andere Firma, \company{New Balance}, setzt ebenfalls auf eine ähnliche Technologien, und will selbst individualisierte Sohlen, welche an den Fuß des Benutzers angepasst werden, an bieten \parencite{NEW_BALANCE}.

\vspace{7pt}

Es ist somit klar ersichtlich dass 3D Druck bald schon in der Industrie eine große Rolle spielen wird. Viele komplexe Bauteile können sehr leicht hergestellt werden, Einzelteile können auf Knopfdruck fabriziert werden, und Individualisierungen von Produkten werden nicht mehr sein als eine kleine Eingabe am Computer. Es wird eine deutliche Revolutionierung des Entwicklungs-, aber auch des Herstellungsprozesses geben, welche sich stark auf die Qualität der Teile und Systeme auswirken wird.
\subsection{Medizin}
\subsection{Privatgebrauch}
