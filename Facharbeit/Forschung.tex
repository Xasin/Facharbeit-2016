3D Drucker sind bisher kaum an die Grenzen des Möglichen gestoßen. Durch Patentrechte und fehlendes Interesse der Industrie an der Entwicklung von Druckern wurden viele Technologien noch kaum erforscht. Dies jedoch beginnt sich langsam zu ändern: Die Wichtigkeit des 3D Druckes steigt rapide an, und dementsprechend auch damit verbunden die Geschwindigkeit der Erforschung neuer Systeme. Was für neue Technologien in letzter Zeit entwickelt wurden bzw. woran momentan gearbeitet wird, soll hier genauer erläutert werden.

\subsection{Multimaterialdruck}
Etwas, das bis heute mit herkömmlichen Verfahren nicht möglich ist, ist die Produktion eines in einem Fertigungsdurchgang hergestelltem Bauteiles aus mehreren Materialien, wie z.~B. festen, biegsamen, leitfähigen oder durchsichtigen Stoffen. Dies jedoch könnte bald schon mit dem \emph{ElectroUV3D} von \company{ChemCubed} möglich werden. Dieser neue Drucker ist mithilfe eines MJM-Verfahrens mit verschiedenen Ausgangsmaterialien in der Lage, komplexe Produkte herzustellen. 

Ein Beispiel einer (sehr simplen) Anwendungsmöglichkeit zeigte die Firma mithilfe eines LED-Lichtes: Sie konnten die Unterseite aus einem Stück drucken, wobei sie aus einem festen Stoff mit in die Platte eingesetzten Leiterbahnen gefertigt wurde. Nachdem LED und Batterie eingesetzt wurden konnte der Drucker eine biegsame Oberseite aufdrucken, wobei auch hier Leiterbahnen eingesetzt wurden. Das Endergebnis war eine in einem Durchgang hergestellte und komplett funktionsfähige Taschenlampe.

Solche Multimaterial-3D Drucker könnten schon bald in der Produktion von komplexen elektronischen Teilen oder speziellen Bauteilen mit verschiedenen, selektiven Eigenschaften (z.~B. biegsame Elektronik) Anwendung finden, da vergleichbare Ergebnisse nur mit mehreren komplexen Produktionsschritten realisierbar sind \parencite{Multimaterial}.

\subsection{Metall-Jetting}
Metall-3D Druck ist zwar heutzutage möglich, benötigt allerdings gesundheitsschädliche extrem feine Metallpulver, und die erstellten Modelle müssen oftmals nach bearbeitet werden. Aus diesem Grund hat sich die Firma \emph{\textquotedblleft XJet\textquotedblright} dazu entschieden den Prozess zu vereinfachen. 

Anstelle eines feinen Metallpulvers, welche mithilfe eines Lasers geschmolzen wird, funktioniert das XJet-System mehr wie ein MJM Drucker: Eine spezielle, mit Metall angereicherte Flüssigkeit wird schichtweise aufgetragen. Hierdurch ist eine hohe Präzision möglich, ohne dabei Material unnötig zu verbrauchen und ohne, dass schädliche feine Metallpulver nach dem Druckvorgang entsorgt werden müssen.

Ähnlich wie die Mikro-3D Druck-Teile können diese so hergestellten feinen Modelle vor allem in der Medizin Verwendung finden, könnten allerdings auch in der generellen Produktion von feinen Bauteilen für Verschiedenes eine hohe Wichtigkeit besitzen \parencite{XJET}.

\subsection{Mikro-3D Druck}
Momentan besitzen viele 3D Drucker nur eine begrenzte Präzision, Oberflächen oder feine Details können meist nicht perfekt abgebildet werden. Oftmals sind allerdings bestimmte feine Details wichtig und müssen so präzise wie möglich hergestellt werden. Um dies zu erlauben hat die \textquotedblleft Vienna University of Technology\textquotedblright ~ein System entwickelt, welches Details im Bereich von wenigen Mikrometern erlaubt.
Das System funktioniert mithilfe eines speziellen SLS-Polymeres, welches von zwei Lasern gleichzeitig getroffen werden muss um aus zu härten. Hierdurch kann um ein Vielfaches besser kontrolliert werden an welchen Stellen das Polymer aushärtet.

Verwendung findet das Verfahren hauptsächlich in der Biomedizin, wo es benutzt werden kann um mikroskopisch kleine Implantate zu drucken, welche z.~B. beschädigtem Gewebe beim Wachstum helfen könnten \parencite{MICROPRINT}.

\subsection{Hochtemperatur-Keramik}
Viele 3D gedruckte Modelle bestehen aus Materialien mit niedrigem Schmelzpunk, z.B. Plastik oder weichen Metallen. Dies wirft für bestimmte Anwendungsgebiete Probleme auf, da die Modelle teils höheren Temperaturen standhalten müssen. Zur Lösung dieses Problems hat nun letztens die Firma \textquotedblleft HRL Labratories\textquotedblright ~ein neues Verfahren zur Herstellung temperaturbeständiger Keramiken bekannt gegeben. Die Modelle werden mithilfe eines Prä-Keramik-Polymeres in einem SLS-ähnlichen Verfahren erstellt, müssen allerdings nach der Fertigung in einen Ofen gebacken zu werden.

Die so gedruckten Modelle sind bis zu 1 400 \TEMP beständig, wodurch sie resistenter sind als die meisten Metalle, und sind zudem leichter und verbiegen sich nicht bzw. werden bei hohen Temperaturen nicht weich.
Dieses Material könnte, durch die hohe Stabilität auch bei hohen Temperaturen, vor allem in der Luft- und Raumfahrt wichtige Applikationen haben, da die Teile auch bei extremer Belastung noch zuverlässig arbeiten müssen \parencite{HiTempCeram}.



\subsection{Gedruckte Medizin}

Ein überraschendes Anwendungsgebiet für 3D Drucker wäre die Herstellung spezieller Tabletten für die besonders schnelle Einnahme lebensrettender Chemikalien. Die meisten momentan verwendeten Tabletten müssen entweder aufgelöst oder im ganzen geschluckt werden, beides kann Zeit brauchen oder ist in einer Notfallsituation nicht leicht möglich. Um dieses Problem zu umgehen hat die pharmazeutische Firma \textquotedblleft Aprecia\textquotedblright ~nun ein neues Konzept entwickelt, mithilfe der schnell lösliche Medizin mit hoher Wirkstoffkonzentration hergestellt werden kann.

Die Pillen werden ähnlich wie beim SLS Verfahren mithilfe eines feinen Pulvers aufgebaut, jedoch werden die \textquotedblleft Spritam levetiracetam\textquotedblright ~genannten Tabletten nicht mit einem Laser sondern mit einem Tropfen Flüssigkeit zusammen gebunden. Das entstehende Produkt ist eine genau abgemessene und vor allem poröse Tablette. Durch diese poröse Struktur löst sich die Pille beinahe vollständig in nur wenigen Sekunden und mit wenig Flüssigkeit auf, und kann dadurch eine hohe Dosis des Wirkstoffes schnell an den Patienten liefern \parencite{Printed_Drug}.

\subsection{10-Material Biodrucker}

Vermutlich eines der wichtigsten Forschungsgebiete des 3D Druckes ist der Bio-3D Druck. In Zukunft könnten bald schon Spenderorgane mithilfe spezieller Systeme von eigenen Zellen gedruckt werden. Dies würde eine Spendersuche unnötig machen, Risiken minimieren und Kosten senken. Diesem Ziel ist die Firma \company{Ourobotics} mit ihrem neuen 10-Material 3D Biodrucker um ein Vielfaches näher gekommen.

Ihr neuer Drucker verwendet nicht eine fest angebrachte Material-Patrone, sondern kann autonom zwischen einer Vielzahl von Patronen mit verschiedenen Materialien wechseln. Zudem besitzt der Drucker keine einfache Druckfläche, sondern einen beheizten Wassertank, welcher In-Vitro Zustände simulieren kann um somit die Lebensdauer der gedruckten Zellen zu verlängern. Zudem arbeitet der Drucker mit einem Gel-Material, welches so gut wie jede Art von biologischem Material enthalten könnte. Somit könnte schon bald simples Gewebe wie z.~B. Muskeln hergestellt werden, theoretisch sogar ganze Organe \parencite{10_BIOPRINT}.

\subsection{Autonome Fabrikation}
Ein wichtiges Problem beim Einsatz von 3D Druckern in die Industrie ist die Autonomie der Maschinen. Die meisten momentanen 3D Drucker müssen regelmäßig von Fachpersonal überprüft, kontrolliert, gereinigt etc. werden, welches einen kompletten Einbau in die Serienproduktion unwirtschaftlich macht. Aus diesem Grund hat sich Firma \company{MetalFAB} an die Entwicklung eines autonomen Metalldruckers gesetzt. 

Ihr momentanes Design, der \emph{MetalFAB1} soll laut Angaben des Herstellers zehn mal zuverlässiger, flexibler und produktiver sein. Der Drucker ist dabei so aufgebaut dass alles, vom anfänglichen SLS bis hin zum Brennen des Modells, vom System automatisch übernommen wird. Zusätzlich besitzt der Drucker mit 420x420x400mm eine der größten Bauflächen der momentan verfügbaren Systeme, und kann dementsprechend große Teile herstellen. Ebenfalls besitzt er vier anstatt nur einen Laser, wodurch die Druckgeschwindigkeit um ein Vielfaches gesteigert wird.
Dieses System soll so den Einbau eines 3D Druckers in die Produktionskette der Industrie, vor allem für komplexe oder individuelle Teile, wirtschaftlich machen \parencite{MetalFAB}.