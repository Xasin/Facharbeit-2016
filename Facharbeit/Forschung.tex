3D Drucker sind bisher kaum an die Grenzen des Möglichen gestoßen. Durch Patentrechte und fehlendes Interesse der Industrie an der Entwicklung von Druckern wurden viele Technologien noch kaum erforscht. Dies jedoch beginnt sich langsam zu ändern: Die Wichtigkeit des 3D Druckes steigt rapide an, und dementsprechend auch damit verbunden die Geschwindigkeit der Erforschung neuer Systeme. Was für neue Technologien in letzter Zeit entwickelt wurden, und woran momentan gearbeitet wird, soll hier genauer erläutert werden.

\subsection{Neue Technologien im 3D Druck}
In letzter Zeit wurde eine Vielzahl neuer, wichtiger Technologien entwickelt. Diese sind zwar noch unausgereift und teuer, könnten allerdings in naher Zukunft ebenso wie die jetzigen Verfahren eine wichtige Rolle innerhalb spezieller Branchen der Industrie spielen, in denen vergleichbare Ergebnisse mit anderen Methoden nur schwer erzielbar sind. Aus diesem Grund wird hier eine kurze Übersicht über einige dieser neuen Technologien gegeben, und wo sie zum Einsatz kommen könnten.

\subsubsection{Hochtemperatur-Keramik}
Viele 3D gedruckte Modelle bestehen aus Materialien mit niedrigem Schmelzpunk, z.B. Plastik oder weichen Metallen. Dies wirft für bestimmte Anwendungsgebiete Probleme auf, da die Modelle teils höheren Temperaturen standhalten müssen. Zur Lösung dieses Problems hat nun letztens die Firma \textquotedblleft HRL Labratories\textquotedblright ~ein neues Verfahren zur Herstellung Temperatur-beständiger Keramiken bekannt gegeben. Die Modelle werden mithilfe eines Prä-Keramik-Polymeres in einem SLS-ähnlichen Verfahren erstellt, müssen allerdings nach der Fertigung in einen Ofen um gebacken zu werden.

Die so gedruckten Modelle sind bis zu 1 400 \TEMP beständig, wodurch sie resistenter sind als die meisten Metalle, und sind zudem leichter und verbiegen sich nicht bzw. werden bei hohen Temperaturen nicht weich.. 
Dieses Material könnte, durch die hohe Stabilität auch bei hohen Temperaturen, vor allem in der Luft- und Raumfahrt wichtige Applikationen haben, bei dem Teile auch bei extremer Belastung noch zuverlässig arbeiten müssen \parencite{HiTempCeram}.

\subsubsection{Mikro-3D Druck}
Momentan besitzen viele 3D Drucker nur eine begrenzte Präzision, Oberflächen oder feine Details können meist nicht perfekt abgebildet werden. Oftmals sind allerdings bestimmte feine Details wichtig und müssen so präzise wie möglich hergestellt werden. Um dies zu erlauben hat die \textquotedblleft Vienna University of Technology\textquotedblright ~ein System entwickelt, welches Details im Bereich von wenigen Mikrometern erlaubt.
Das System funktioniert mithilfe eines speziellen SLS-Polymeres, welches von zwei Lasern gleichzeitig getroffen werden muss um aus zu härten. Hierdurch kann um ein Vielfaches besser kontrolliert werden an welchen Stellen das Polymer aushärtet.

Verwendung findet das Verfahren hauptsächlich in der Biomedizin, wo es benutzt werden kann um mikroskopisch kleine Implantate zu drucken, welche z.~B. beschädigtem Gewebe beim Wachstum helfen könnte \parencite{MICROPRINT}.

\subsubsection{Metall-Jetting}
Metall-3D Druck ist zwar heutzutage möglich, benötigt allerdings gesundheitsschädliche extrem feine Metallpulver, und die erstellten Modelle müssen oftmals nach bearbeitet werden. Aus diesem Grund hat sich die Firma \emph{\textquotedblleft XJet\textquotedblright} dazu entschieden den Prozess zu vereinfachen. 

Anstelle eines feinen Metallpulvers, welche mithilfe eines Lasers geschmolzen wird, funktioniert das XJet-System mehr wie ein MJM Drucker: Eine spezielle, mit Metall angereicherte Flüssigkeit wird schichtweise aufgetragen. Hierdurch ist eine hohe Präzision möglich, ohne dabei Material unnötig zu verbrauchen, und ohne dass schädliche feine Metallpulver nach dem Druckvorgang entsorgt werden müssen.

Ähnlich wie die Mikro-3D Druck-Teile können diese so hergestellten feinen Modelle vor allem in der Medizin Verwendung finden, könnten allerdings auch in der generellen Produktion von feinen Bauteilen für Verschiedenes eine hohe Wichtigkeit besitzen \parencite{XJET}.


\subsection{Pläne und Ziele}
3D Drucker werden schon in naher Zukunft vielseitig zum Einsatz kommen, und viele Firmen besitzen bereits Pläne, wie die Technologien verwendet werden können. Was genau an Planungen momentan vor liegt soll hier im Groben betrachtet werden.

\subsubsection{Hitzebeständige Bauteile}

Die bereits genannten starken und hitzebeständigen Keramik-Bauteile werden eine wichtige Rolle in der Luft- und Raumfahrt dar stellen. Keramiken werden dort oftmals benutzt um vor Hitze ab zu schirmen, so z.~B. innerhalb eines Triebwerkes, oder beim Wiedereintritt in die Atmosphäre. Keramiken haben allerdings einen entscheidenden Nachteil: Sie sind zerbrechlich, und können nur schwer in die gewünschte Form gebracht werden. Sie können nicht gegossen werden, und einmal ausgehärtet ist es nur schwer möglich die Form zu ändern. Zudem zerbrechen sie während der Benutzung, und müssen dadurch oftmals ausgetauscht werden.

Hier kommen nun die 3D Druckbaren Keramiken ins Spiel: Sie können direkt und ohne großen Aufwand in jegliche Form gebracht werden, ohne dabei an Stabilität zu verlieren. So können nun mit Keramik hoch komplexe Teile hergestellt werden, welche stabiler oder leichter sein können als herkömmlich hergestellte Materialien. Diese neuen Keramik-Teile könnten bald schon unersetzlich für die Herstellung von Flugzeugen, Turbinen, Triebwerken, Raumschiffen und vielen anderen Systemen mit hohen Temperaturbelastungen sein, da sie eine mit Metall vergleichbare Stabilität haben, ohne bei hohen Temperaturen an Festigkeit zu verlieren, und sind dabei zudem leichter \parencite{SPACECeramics}.

\subsubsection{3D Gedruckte Medizin}
