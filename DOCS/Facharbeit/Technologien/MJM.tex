\subsubsection{MultiJet Modeling (MJM)}

Das MultiJet-Modeling kann als Kombination von SLA und FDM angesehen werden. Es arbeitet mit einem anfangs festen, wachsähnlichem Material, welches wie beim FDM geschmolzen und danach auf die vorherige Schicht aufgetragen wird. Allerdings wird das Plastik nun, wie beim SLA, mithilfe einer UV-Lampe gehärtet um die Stabilität zu erhöhen. Somit können Teile mit hoher Präzision schnell und kostengünstig hergestellt werden. Die meisten 3D Drucker mit MJM Verfahren sind sehr teuer, und sind dadurch nur für größere Unternehmen lohnenswert.

Wie bereist erwähnt ist MJM eine Kombination aus sowohl FDM und SLA. Ein festes, schmelzbares Thermoplast oder Hartwachs wird im Druckkopf in der gewünschten Menge geschmolzen. Nun wird das flüssige Material mithilfe von mehreren Düsen auf die vorherige Schicht aufgetragen und mit einer kleinen Walze zu einer einheitlichen Schicht geglättet. Diese neue Schicht wird anschließend mit einer starken UV-Lampe gehärtet. Wie auch bei den anderen Verfahren wird dies Schichtweise durchgeführt bis das Endmodell fertig ist. Durch Verwendung mehrerer Materialien mit unterschiedlichen Schmelzpunkten (z.B. einem Wachs) ist es sogar möglich, Stützstrukturen zu drucken, welche sich durch leichtes Aufwärmen erneut verflüssigen. Hierdurch kann die Produktion eines 3D-Modells komplett automatisiert werden, ohne dass ein menschliches Eingreifen notwendig ist.

Sehr vorteilhaft an diesem Verfahren ist die hohe Geschwindigkeit, mit der Modelle gedruckt werden können. Da der Druckkopf mehrere Düsen besitzt kann mehr Material gleichzeitig aufgetragen werden, ohne dabei die Qualität des Modells zu beeinflussen. Auch können die Modelle dank sehr feiner Düsen mit sehr hoher Präzision gedruckt werden. Da die Stützstrukturen aus einem wachsähnlichen Material gefertigt werden, können diese durch leichtes Erwärmen entfernt werden. Theoretisch könnte auch das gesamte Modell aus Wachs gedruckt werden, und eignet sich dadurch als Gießmodell für ein Metallobjekt.

Einziger Nachteil dieser Technologie ist der hohe Preis bzw. die geringe Verfügbarkeit der Materialien und Drucker, welche momentan nur von einigen wenigen Firmen hergestellt werden \parencite[Informationen aus:][]{DRUCKVERFAHREN,WIKIMJM}.