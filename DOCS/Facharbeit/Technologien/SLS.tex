\subsubsection{Selektives Lasersintern (SLS), Selektives Laserschmelzen (SLM) und Selektives Elektronenstrahlschmelzen (SEBM)}

Alle drei aufgeführten Verfahren arbeiten auf eine ähnliche Funktionsweise, bei der ein pulverförmiger Feststoff mithilfe eines (Laser)Strahls zusammengeschmolzen wird. Aus diesem Grund wird hier nur das Selektive Lasersintern (SLS) beschrieben.

Beim SLS wird zuerst ein pulverförmiges Material dünn auf entweder die Basisplatte oder die vorherige Schicht aufgetragen und mithilfe einer Walze geglättet. Hierbei kann das Material theoretisch alles sein, sofern es als feines Pulver vorliegt und der Schmelzpunkt erreicht werden kann. Nun wird das Pulver an den gewollten stellen mithilfe eines starken Lasers geschmolzen (SLM) oder gesintert (SLS, d.H. ein Bindemittel bzw die Oberfläche des Pulvers wird angeschmolzen um die Pulverkörner miteinander zu verkleben). Vorteilhaft beim schmelzen des gesamten Pulvers ist ein festeres Endmodell, dies erfordert jedoch auch eine höhere Temperatur wodurch der Druckvorgang verlangsamt wird. Dieser Vorgang wird nun mit jeder Schicht wiederholt, wobei sich die individuellen Schichten ebenfalls miteinander verbinden, bis das fertige Modell entstanden ist. Will man das Objekt entnehmen so muss man vorher das Pulver entfernen. Dies geschieht meistens vom Drucker, jedoch ein feiner Staub zurück bleiben welcher selbst noch entfernt werden muss. In industriellen Applikationen befindet sich der Drucker meist in einer Vakuumkammer. Dies verhindert Verunreinigungen durch sich in der Luft befindlichen Partikel und sorgt für eine höhere Reinheit des Modells.

Vorteilhaft an dieser Methodik ist die Verwendbarkeit von quasi jedem pulverförmigem Material. Mithilfe von SLS bzw. SLM können Plastik, Metalle und sogar Keramik problemlos gedruckt werden. Auch die Qualität ist eine der höchsten aller Verfahren, mit Schichtdicken von nur 20 Mikrometern. Auch müssen keinerlei Stützstrukturen verwendet werden, da das gesamte Modell vom Pulver umgeben und gestützt wird. Durch das Zusammenschmelzen des Stoffes ist bei Metallen das Endmodell in der Stabilität vergleichbar mit gegossenem Metall, und bei Kunststoffen können während des Druckens Farben hinzugegeben werden, um beliebig eingefärbte Drucke zu produzieren.

Allerdings hat dieses Verfahren auch seine Nachteile. Das Pulver welches verwendet wird ist durch seine Feinheit gefährlich für Menschen und es muss sehr vorsichtig mit dem Material umgegangen werden. Zusätzlich sorgt dieses Pulver auch für eine umständliche Entsorgung, welche mit teuren Filteranlagen realisiert werden muss. All dies sorgt dafür dass diese Verfahren eine der komplexesten und teuersten sind, und deshalb nur für Industrie nützlich sind. \parencite[Informationen aus:][]{DRUCKVERFAHREN,SLSDetail}