\subsubsection{Fused Deposition Modeling (FDM):}

FDM ist die momentan am weitesten bekannte und für Privatpersonen oder Kleinunternehmen geeignetste 3D Druck-Methode. Durch das sehr simple Verfahren können auch schon mit billigen 3D Druckern im Wert von etwa 700\EURO ~ (bei Selbstbau-Modellen teils weniger) gute Ergebnisse erzielt werden, jedoch gibt es auch einige Nachteile des FDM.

Bei dieser Variante des 3D Druck wird ein Plastik-Filament, meistens PLA oder ABS, mithilfe eines Motors durch eine beheizte Düse gepresst. Das geschmolzene Plastik wird nun auf die Druckunterlage bzw. das bereits Gedruckte in der gewünschten Form aufgetragen und erstarrt dort.
Dies wird schichtweise so lange wiederholt bis das endgültige Modell fertig ist. Gedruckt wird hierbei meist nur ein, bei einigen Modellen mit mehreren Düsen zwei oder mehr Material, welches als Filament-Strang von einer Rolle abgewickelt werden kann.

Das verhältnismäßig simple Verfahren erlaubt für den Bau kostengünstiger 3D Drucker welche oftmals in Selbstbauweise gebaut, repariert und gewartet werden können. Beispiele für diese \textquotedblleft Selbstbau-3D-Drucker\textquotedblright~kann man auf der Seite \citeurl{REPRAP} finden.

Vorteilhaft am FDM Verfahren ist ebenfalls eine breite Menge an kostengünstigen Materialien, welche sich zum Drucken eignen. So kann z.~B. mit PLA, ABS, HIPS, PVA, Nylon, PET, PETT, PC und TPE \parencite{MATERIALS} ohne Umbau oder Änderung der Hardware gearbeitet werden, lediglich die Temperatur muss angepasst werden.

Jedoch hat FDM auch seine Nachteile.
Durch das schichtweise aufbauende Verfahren hat das Modell klar getrennte sichtbare \textquotedblleft Schichtungen", und zudem werden dadurch kleinere Details womöglich nicht korrekt wiedergegeben. Ebenso dauert der Druck durch die wiederholten Bewegungen des schweren Druckkopfes vor allem bei hoher Genauigkeit sehr lange. Viele Drucker brauchen eine korrekte Feinjustierung der Einzelteile bevor brauchbare Stücke gedruckt werden können. Außerdem gibt es einige materialbedingte Probleme. So können sich Bauteile während oder nach des Druckens durch unterschiedlich schnelles Abkühlen verziehen. Will man z.~B. eine steile Kante oder einen sog. Überhang, also eine Stelle ohne Stützen oder darunterliegendes Material drucken, so kann es oftmals dazu kommen, dass das noch flüssige Material an dieser Stelle herunter hängt oder der Druck ohne zusätzliches Stützmaterial, welches nach dem Druck von Hand entfernt werden muss, nicht möglich ist \parencite[Informationen aus:][]{FDMDetail,DRUCKVERFAHREN}.