
Der 3D Druck ist erst seit einigen Jahren als Konzept bekannt geworden.
Zwar existierte das Prinzip des 3D Druckens schon seit mehreren Jahren, um genauer zu sein seit 1983 \parencite{CHUCK}, jedoch waren Verwendungsmöglichkeiten und Forschung wegen eines im Jahre 1986 erstellten Patentes bis vor kurzem weitgehend eingeschränkt.

Seit einigen Jahren arbeiten Industrie und Forschung an immer besser werdenden, schneller arbeitenden 3D Druckern. In diesem Kapitel wird aufgeführt, wie die jetzige Situation der Technologie aussieht, und wo und wie sie verwendet wird.

\subsection{Aktueller technischer Stand}

Im 3D Druck kann auf sehr viele verschiedene Arten und Weisen ein Modell gedruckt werden und auch die einzelnen Verfahren an sich können sich untereinander stark unterscheiden.
Deshalb soll hier auf die vier\footnote{Persönliche Einschätzung des Autors} wichtigsten Verfahren eingegangen werden:

\subsubsection{Fused Deposition Modeling (FDM):}

FDM ist die momentan am weitesten bekannte und für Privatpersonen oder Kleinunternehmen geeignetste 3D Druck-Methode. Durch das sehr simple Verfahren können auch schon mit billigen 3D Druckern im Wert von etwa 700\EURO ~ (bei Selbstbau-Modellen teils weniger) gute Ergebnisse erzielt werden, jedoch gibt es auch einige Nachteile des FDM.

Bei dieser Variante des 3D Druck wird ein Plastik-Filament, meistens PLA oder ABS, mithilfe eines Motors durch eine beheizte Düse gepresst. Das geschmolzene Plastik wird nun auf die Druckunterlage bzw. das bereits Gedruckte in der gewünschten Form aufgetragen und erstarrt dort.
Dies wird schichtweise so lange wiederholt bis das endgültige Modell fertig ist. Gedruckt wird hierbei meist nur ein, bei einigen Modellen mit mehreren Düsen zwei oder mehr Material, welches als Filament-Strang von einer Rolle abgewickelt werden kann.

Das verhältnismäßig simple Verfahren erlaubt für den Bau kostengünstiger 3D Drucker welche oftmals in Selbstbauweise gebaut, repariert und gewartet werden können. Beispiele für diese \textquotedblleft Selbstbau-3D-Drucker\textquotedblright~kann man auf der Seite \citeurl{REPRAP} finden.

Vorteilhaft am FDM Verfahren ist ebenfalls eine breite Menge an kostengünstigen Materialien, welche sich zum Drucken eignen. So kann z.~B. mit PLA, ABS, HIPS, PVA, Nylon, PET, PETT, PC und TPE \parencite{MATERIALS} ohne Umbau oder Änderung der Hardware gearbeitet werden, lediglich die Temperatur muss angepasst werden.

Jedoch hat FDM auch seine Nachteile.
Durch das schichtweise aufbauende Verfahren hat das Modell klar getrennte sichtbare \textquotedblleft Schichtungen", und zudem werden dadurch kleinere Details womöglich nicht korrekt wiedergegeben. Ebenso dauert der Druck durch die wiederholten Bewegungen des schweren Druckkopfes vor allem bei hoher Genauigkeit sehr lange. Viele Drucker brauchen eine korrekte Feinjustierung der Einzelteile bevor brauchbare Stücke gedruckt werden können. Außerdem gibt es einige materialbedingte Probleme. So können sich Bauteile während oder nach des Druckens durch unterschiedlich schnelles Abkühlen verziehen. Will man z.~B. eine steile Kante oder einen sog. Überhang, also eine Stelle ohne Stützen oder darunterliegendes Material drucken, so kann es oftmals dazu kommen, dass das noch flüssige Material an dieser Stelle herunter hängt oder der Druck ohne zusätzliches Stützmaterial, welches nach dem Druck von Hand entfernt werden muss, nicht möglich ist \parencite[Informationen aus:][]{FDMDetail,DRUCKVERFAHREN}.
\subsubsection{Stereolithographie (SLA)}

Die Stereolithographie ist eine alternative Technologie zur Erstellung von 3D Modellen, welche mit hohem Detail und hoher Geschwindigkeit arbeiten kann. Jedoch sind die Modelle lichtempfindlich, und das Ausgangsmaterial ist um ein vielfaches teurer als bei anderen Varianten.

Beim SLA Verfahren wird, anders als bei anderen Möglichkeiten, mit einem bei Raumtemperatur flüssigem Kunstharz gearbeitet, welches sich in einem Becken befinden.
Dieses Kunstharz wird nun mithilfe eines starken UV-Lasers an den gewollten Stellen gehärtet, wobei ähnlich wie beim FDM Verfahren schichtweise gearbeitet wird.

Ist eine Schicht des Objektes fertig, so wird entweder der Träger auf dem sich das Modell befindet ein Stück weiter aus dem Kunstharz heraus gezogen (Hierbei trifft der Laser von unten durch eine durchsichtige Platte auf das Kunstharz) oder eine kleine Menge Harz hinzu gegeben, sodass das Modell konstant mit einer Schicht dieses Materials bedeckt ist (Hierbei trifft der Laser von oben auf das Harz).

Ist das Modell fertig kann es aus dem Harz entnommen werden, muss allerdings oftmals noch mithilfe eines UV-Schrankes weiter gehärtet werden, um die Stabilität des Modells zu gewährleisten.

Sehr vorteilhaft ist hierbei dass die Modelle mit hoher Präzision gedruckt werden können da der Laser quasi unbegrenzt fein sein kann, wodurch kaum noch das Objekt nachbearbeitet werden muss. Höchstens die für Überhänge nötigen Stützstrukturen müssen entfernt werden, diese können jedoch oftmals um ein Vielfaches kleiner sein da das Modell während des Druckes vom Kunstharz umgeben ist welches die gleiche Dichte besitzt, und so \textquotedblleft in der Schwebe\textquotedblright ~gehalten wird.

Nachteile des SLA sind allerdings vor allem hohe Kosten des Kunstharzes, sowie die verbleibende Lichtempfindlichkeit des Modells, welche bei übermäßigem Lichteinfall zu Sprödigkeit führen kann. \parencite[Informationen aus:][]{DRUCKVERFAHREN}
\subsubsection{MultiJet Modeling (MJM)}

Das MultiJet-Modeling kann als Kombination von SLA und FDM angesehen werden. Es arbeitet mit einem anfangs festen, wachsähnlichem Material, welches wie beim FDM geschmolzen und danach auf die vorherige Schicht aufgetragen wird. Allerdings wird das Plastik nun, wie beim SLA, mithilfe einer UV-Lampe gehärtet um die Stabilität zu erhöhen. Somit können Teile mit hoher Präzision schnell und kostengünstig hergestellt werden. Die meisten 3D Drucker mit MJM Verfahren sind sehr teuer, und sind dadurch nur für größere Unternehmen lohnenswert.

Wie bereist erwähnt ist MJM eine Kombination aus sowohl FDM und SLA. Ein festes, schmelzbares Thermoplast oder Hartwachs wird im Druckkopf in der gewünschten Menge geschmolzen. Nun wird das flüssige Material mithilfe von mehreren Düsen auf die vorherige Schicht aufgetragen und mit einer kleinen Walze zu einer einheitlichen Schicht geglättet. Diese neue Schicht wird anschließend mit einer starken UV-Lampe gehärtet. Wie auch bei den anderen Verfahren wird dies Schichtweise durchgeführt bis das Endmodell fertig ist. Durch Verwendung mehrerer Materialien mit unterschiedlichen Schmelzpunkten (z.B. einem Wachs) ist es sogar möglich, Stützstrukturen zu drucken, welche sich durch leichtes Aufwärmen erneut verflüssigen. Hierdurch kann die Produktion eines 3D-Modells komplett automatisiert werden, ohne dass ein menschliches Eingreifen notwendig ist.

Sehr vorteilhaft an diesem Verfahren ist die hohe Geschwindigkeit, mit der Modelle gedruckt werden können. Da der Druckkopf mehrere Düsen besitzt kann mehr Material gleichzeitig aufgetragen werden, ohne dabei die Qualität des Modells zu beeinflussen. Auch können die Modelle dank sehr feiner Düsen mit sehr hoher Präzision gedruckt werden. Da die Stützstrukturen aus einem wachsähnlichen Material gefertigt werden, können diese durch leichtes Erwärmen entfernt werden. Theoretisch könnte auch das gesamte Modell aus Wachs gedruckt werden, und eignet sich dadurch als Gießmodell für ein Metallobjekt.

Einziger Nachteil dieser Technologie ist der hohe Preis bzw. die geringe Verfügbarkeit der Materialien und Drucker, welche momentan nur von einigen wenigen Firmen hergestellt werden \parencite[Informationen aus:][]{DRUCKVERFAHREN,WIKIMJM}.
\subsubsection{Selektives Lasersintern (SLS), Selektives Laserschmelzen (SLM) und Selektives Elektronenstrahlschmelzen (SEBM)}

Alle drei aufgeführten Verfahren arbeiten auf eine ähnliche Funktionsweise, bei der ein pulverförmiger Feststoff mithilfe eines (Laser)Strahls zusammengeschmolzen wird. Aus diesem Grund wird hier nur das Selektive Lasersintern (SLS) beschrieben.

Beim SLS wird zuerst ein pulverförmiges Material dünn auf entweder die Basisplatte oder die vorherige Schicht aufgetragen und mithilfe einer Walze geglättet. Hierbei kann das Material theoretisch alles sein, sofern es als feines Pulver vorliegt und der Schmelzpunkt erreicht werden kann. Nun wird das Pulver an den gewollten stellen mithilfe eines starken Lasers geschmolzen (SLM) oder gesintert (SLS, d.H. ein Bindemittel bzw die Oberfläche des Pulvers wird angeschmolzen um die Pulverkörner miteinander zu verkleben). Vorteilhaft beim schmelzen des gesamten Pulvers ist ein festeres Endmodell, dies erfordert jedoch auch eine höhere Temperatur wodurch der Druckvorgang verlangsamt wird. Dieser Vorgang wird nun mit jeder Schicht wiederholt, wobei sich die individuellen Schichten ebenfalls miteinander verbinden, bis das fertige Modell entstanden ist. Will man das Objekt entnehmen so muss man vorher das Pulver entfernen. Dies geschieht meistens vom Drucker, jedoch ein feiner Staub zurück bleiben welcher selbst noch entfernt werden muss. In industriellen Applikationen befindet sich der Drucker meist in einer Vakuumkammer. Dies verhindert Verunreinigungen durch sich in der Luft befindlichen Partikel und sorgt für eine höhere Reinheit des Modells.

Vorteilhaft an dieser Methodik ist die Verwendbarkeit von quasi jedem pulverförmigem Material. Mithilfe von SLS bzw. SLM können Plastik, Metalle und sogar Keramik problemlos gedruckt werden. Auch die Qualität ist eine der höchsten aller Verfahren, mit Schichtdicken von nur 20 Mikrometern. Auch müssen keinerlei Stützstrukturen verwendet werden, da das gesamte Modell vom Pulver umgeben und gestützt wird. Durch das Zusammenschmelzen des Stoffes ist bei Metallen das Endmodell in der Stabilität vergleichbar mit gegossenem Metall, und bei Kunststoffen können während des Druckens Farben hinzugegeben werden, um beliebig eingefärbte Drucke zu produzieren.

Allerdings hat dieses Verfahren auch seine Nachteile. Das Pulver welches verwendet wird ist durch seine Feinheit gefährlich für Menschen und es muss sehr vorsichtig mit dem Material umgegangen werden. Zusätzlich sorgt dieses Pulver auch für eine umständliche Entsorgung, welche mit teuren Filteranlagen realisiert werden muss. All dies sorgt dafür dass diese Verfahren eine der komplexesten und teuersten sind, und deshalb nur für Industrie nützlich sind. \parencite[Informationen aus:][]{DRUCKVERFAHREN,SLSDetail}

\subsection{3D Druck in der Industrie}

Bis jetzt war der 3D Druck nur wenig in der eigentlichen Industrie vertreten. Oftmals werden die langsamen Geräte zum Rapid Prototying verwendet da sie kostengünstiger neue Prototypen herstellen können als andere Verfahren, finden allerdings nur selten ihren Weg in die Produktion größerer Konzerne. Jedoch finden vor allem SLS 3D Drucker ihren Platz in der Herstellung hoch komplexer Teile, die mit herkömmlichen Verfahren zu komplex gewesen wären. 

Bestes Beispiel dieser 'exzentrischen' Nische in der sich 3D Drucker wiedergefunden haben sind die Kooperationen zwischen Airbus und Rolls-Royce, sowie der 3D Druck Firma Stratasys: Der Großkonzern will über 1.000 Teile des Airbus A350 mithilfe von 3D Druckern herstellen lassen. 2015 wurde das erste Triebwerk mit einem der größten 3D gedruckten Teilen jemals erfolgreich getestet und geflogen. Das 3D gedruckte Frontlagergehäuse wurde mithilfe eines SEBM Druckers aus Titan gefertigt, war 1.5m im Durchmesser, und reduzierte die Produktionszeit um beinahe 30\%, während zusätzlich Kosten und Funktion des Teiles optimiert wurden. Dieses Laufwerk, die \textquotedblleft Trent XWB-97\textquotedblright sei mit ca. 44 000kg Schubkraft eine der stärksten Flugzeugtriebwerke, welche jemals hergestellt wurden, laut \textcite{TRENT}. Das Triebwerk soll schon 2017 serienmäßig in den A350-1000 eingebaut werden, und wurde bisher an 41 Käufer verkauft, womit es eine der meist verkauften Triebwerke bisher gilt.

3D Druck findet sich allerdings nicht nur in der Luftfahrt. Der Fahrzeughersteller \emph{\textquotedblleft Local Motors\textquotedblright} hat sich dafür entschieden, mithilfe von 3D Druck den Entwicklungsprozess neuer Autos von 6 Jahren auf nur 4 Monate herunter zu bringen. 3D Drucker würden hierbei eine schnelle Änderung der Form, des Materials oder anderer Eigenschaften erlauben, welches Produktionszeiten eines Prototypen signifikant senken könnten \parencite{Local_Motor}. 
Diese eindeutige Verbesserung der Entwicklungszeit durch 3D Drucker findet man auch in anderen Gebieten. So können nun kleine Modelle, welche vorher 500\EURO kosteten, durch eine externe Firma gefertigt wurden und erst nach mehreren Tagen geliefert wurden für weniger als 15\EURO direkt am Arbeitsplatz in nur wenigen Minuten produziert werden \parencite{BALDOR_CASE}.
Diese deutliche Beschleunigung des Prototypingprozesses sorgt für einen schnelleren Entwicklungsprozess insgesamt, und sorgt so im Endeffekt für besser gefertigte und getestete Modelle in einer kürzeren Zeitspanne.

Momentan spielt 3D Druck in der eigentlichen Industrie noch keine Rolle. Anwendung findet er jedoch jetzt schon für die Produktion von komplexen Einzelteilen bzw. Kleinserien, und greift auch jetzt schon in den Produktionsprozess ein, als ein Mittel um die Entwicklung neuer Teile um ein Vielfaches zu beschleunigen.

\subsection{Medizinischer Gebrauch}

Ein Gebiet in dem der 3D Druck auch jetzt schon eine sehr große Bedeutung besitzt ist die Medizin. In kaum einem anderen Anwendungsgebiet ist die Flexibilität eines 3D Druckers so gefragt: jeder Patient hat seine eigenen Ansprüche, jeder Körper ist unterschiedlich, und benötigt deshalb angepasste, individuelle Teile. Bevor es 3D Drucker gab mussten diese entweder teuer produziert und verarbeitet werden, und selbst hier gab es Einschränkungen, die wichtigste davon ist die Zeit in der die Teile gebraucht werden. Mit 3D Druckern jedoch lässt sich so gut wie jede Form in jedem Material drucken, Teile sind in nur wenigen Stunden, nicht Tagen, verfügbar.

\subsubsection{Prothesen}
Beispiele für die Verwendung von 3D Druckern im medizinischen Bereich gibt es jetzt schon. So wurde einem 18-Monate altem Kind, Garrett Peterson, zwei speziell angefertigte gedruckte Schienen für seine Luftröhre eingesetzt. Garrett litt an einer Krankheit welche seine Luftröhre destabilisiert hatte, wodurch diese in sich zusammenfallen konnte. Um diesem entgegenzuwirken entschieden sich seine Ärzte zu einer neuartigen Maßnahme. Sie fertigten ein digitales Modell der Luftröhre an, und entwickelten basierend auf dem Modell zwei spezielle Schienen, welche um sie gelegt werden konnten um sie so offen zu halten. Die Schienen wurden mithilfe eines SLM Druckers und einem speziellen Kunststoff angefertigt. Die Operation verlief erfolgreich, und laut den Ärzten würde Garrett nach wenigen Monaten schon aus dem Krankenhaus entlassen werden können \parencite{BABY}.

\subsubsection{Entwicklung}
3D Druck hat allerdings nicht nur direkt in der Medizin Einfluss. Auch die Produktionskosten medizinischer Teile, vor allem in Entwicklungsländern, wird mithilfe des 3D Druck um ein Vielfaches gesenkt. Wichtige medizinische Versorgung kann so besser für die ärmeren Länder zur Verfügung gestellt werden. So hat sich z.~B. die Firma \emph{\textquotedblleft Cardiac Design Labs\textquotedblright} dazu entschieden ein kostengünstiges, mobiles EEG Gerät für vor allem die armen Gebiete in Indien her zu stellen. Grund dafür ist die schlechte Versorgung der Indischen Bevölkerung, vor allem für Herz- und Kreislaufstörungen. Schätzungen gehen davon aus, dass 26\% aller Tode durch Herzfehler hervorgerufen werden. 3D Druck half hierbei sehr stark bei der schnellen Entwicklung und Produktion der neuen Geräte aus. Die gedruckten Modelle waren weitaus stärker, genauer und fester als die vorher von einer Zweitfirma gelieferten Prototypen, und konnten zudem billiger und schneller hergestellt werden. Dies sorge dafür, dass die Entwickler früher als erwartet in die erste Beta-Testphase gehen konnten. Auch hier halfen 3D Drucker als eine billigere und schnellere Alternative zur Produktion der Einzelteile aus \parencite{MIRCAM_STUDY}.

\subsubsection{Training}
Ein Gebiet, in dem die Vorteile des 3D Druckes ebenfalls gut sichtbar sind, ist die Erstellung von möglichst realistischen Dummys zum Üben und Trainieren von lebensrettenden Maßnahmen. Eine dieser Maßnahmen ist der Umgang mit Patienten, dessen Atmung eingeschränkt oder blockiert ist. 

Realistische, Körper-getreue Modelle sind sehr kompliziert zu produzieren, und dementsprechend schlecht verbreitet. Laut einer Studie im Juni 2012 sei die Anatomie der Trainingsdummys nicht passend geformt, und würde den Aufbau echter Patienten nicht korrekt wiedergeben. Diese Fehler hätten einen negativen Einfluss auf das Training. Abhilfe kann hierbei die Produktion von Dummys mithilfe von 3D Druckern sein. So entschied sich die \emph{\textquotedblleft University of Minnesota Medical School\textquotedblright} dazu, mithilfe von MRI-Scans echter Patienten ein möglichst getreues 3D Modell der einzelnen Organe zu erstellen. Dieses Modell konnte später mithilfe eines 3D Druckers in Einzelteilen hergestellt werden. Sehr hilfreich war dabei die Möglichkeit des Druckers, mit verschiedenen Materialien zu drucken. Dies erlaubte es den Entwicklern, bestimmte Teile wie z.~B. die Wirbelsäule aus verschieden biegsamen Materialien zu fertigen, um die Beweglichkeit besser zu simulieren. 

Das Endergebnis dieses Prozesses ist einer der laut Jack Stubbs, Leiter des Projektes an der Universität, realistischsten Trainingsdummys. Bald soll die Produktion solcher realistischen Dummys in Serie gehen, wobei 3D Druck eine wichtige Rolle im Prozess spielen soll. Die neuen Dummys können so ein akkurates Training ermöglichen, und verbessern damit die Überlebenschancen für Personen mit Atemwegsbeschwerden und -Problemen in Notfallsituationen \parencite{DUMMY_STUDY}.


\subsection{Privatgebrauch}

Eines der Gebiete in dem eine der größten Änderungen vorkommen wird ist der Privatgebrauch, bzw. für Kleinfirmen:
3D Drucker erlauben diesen (Einzel)Personen ein schnelles, kostengünstiges Herstellen eigener Teile. Dies wäre sonst nur mit teuren Geräten und über langsam laufende Prozesse möglich, und wäre dadurch mit sehr hohen Kosten für Einzelteile verbunden. 3D Drucker hingegen können auf Wunsch ein beliebiges Modell in kurzer Zeit und nur mit Strom- und Materialkosten drucken, und sind dadurch für Kleinserien oder Einzelteile um ein vielfaches günstiger. 

Jedoch ist die momentane Technologie noch sehr unausgereift. Hauptsächlich FDM ist kostenmäßig für Einzelpersonen lohnenswert, dies ist jedoch eine der langsamsten Technologien. Zudem ist selbst hierbei der Preis meist noch relativ hoch: Gute Modelle sind ab etwa 500\EURO laut \textcite{TOM_PRINTER} käuflich, für weniger muss man mit Einbußen bei Geschwindigkeit, Druckgröße und Präzision rechnen. Eine Ausweichmöglichkeit wäre der Selbstbau eines Druckers, mit dem Kosten auf ca. 350\EURO bei gleichbleibender Qualität gesenkt werden können. Dies jedoch erfordert Wissen und Erfahrung mit der Technik, da oftmals Fehler auftreten können welche manuell behoben werden müssen\footnote{Informationen bezogen aus eigener Erfahrung}.

Zudem ist der Nutzen für die meisten Menschen kaum groß. Oftmals können bestimmte Dinge aus einem Laden gekauft werden, oder individualisierte Teile werden nur sehr selten benötigt. Sollte nun ein Modell gedruckt werden, so gibt es Dienste, welche dieses Drucken übernehmen können. Zwar ist dies teurer als das Drucken mit eigenen Geräten, man braucht hierfür jedoch selbst keinen Drucker. So lohnt sich für viele Menschen die Anschaffung eines 3D Druckers noch nicht \parencite[Informationen aus:][]{PRIV_USE}.
